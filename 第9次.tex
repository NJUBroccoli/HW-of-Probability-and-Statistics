\documentclass[twocolumn]{article}

\usepackage[utf8]{inputenc}
\usepackage{CJKutf8}
\usepackage{CJK}
\usepackage{algorithm}
\usepackage{algorithmic}
\usepackage{amsmath}
\usepackage{amsthm}
\usepackage{amssymb}
\usepackage{newfloat}
\usepackage{setspace}
\usepackage{tikz}
\usepackage{enumerate}
\usepackage{listings}
\usepackage{fancyhdr}
\allowdisplaybreaks[4]
\usetikzlibrary{arrows,graphs}
\usetikzlibrary{graphs}
\usetikzlibrary{graphs.standard}
\pagestyle{fancy}
\fancyhead[L]{Probability and Mathematical Statistics}
\begin{document}
	\begin{CJK}{UTF8}{gbsn}		
			\title{概率论与数理统计第7、8次作业}
			\author{161220049\quad 黄奕诚}
			\maketitle
			
			\section*{教材习题三}
			\subsection*{13}
			\subsubsection*{(1)}
				先求分布函数$F_Z(z)$.因为$X$和$Y$独立,所以有\[p(x,y)=p(x)\cdot p(y)\]因此当$0<z<1$时,有\begin{align*}
					F_Z(z) & = P(X+Y\le z)\\
					& = \int_{0}^{z}\int_{0}^{z-x}e^{-y}dxdy\\
					& = \int_{0}^{z}dx\int_{0}^{z-x}e^{-y}dy\\
					& = \int_{0}^{z}(1-e^{x-z})dx\\
					& = (x-e^{x-z})|_0^z\\
					& = z-1+e^{-z}
				\end{align*}
				此时\[p_Z(z)=F_Z'(z)=1-e^{-z}\]当$z\ge1$时,有\begin{align*}
					F_Z(z) & = P(X+Y\le z)\\
					& = \int_{0}^{1}dx\int_{0}^{z-x}e^{-y}dy\\
					& = (x-e^{x-z})|_0^1\\
					& = 1-e^{-z}(e-1)
				\end{align*}
				此时\[p_Z(z)=F_Z'(z)=e^{-z}(e-1)\]综上所述,\begin{equation*}
				p_Z(z)=\left\{
				\begin{array}{rcl}
				0 & & {z\le0}\\
				1-e^{-z} & & {0<z<1}\\
				e^{-z}(e-1) & & {z\ge1}\\
				\end{array} \right.
				\end{equation*}
			\subsubsection*{(3)}
				首先需满足$0<x<1$,否则$X=x$事件不可能发生.若$z\le x$,则有$y\le0$,此时有$p_Z(z)=0$.因此只考察$z>x$的情况,此时\begin{align*}
					F_{Z|X=x}(z) & = \int_{0}^{z-x}e^{-y}dy\\
					& = -(e^{x-z}-1)\\
					& = 1-e^{x-z}
				\end{align*}
				求导则有\[p_{Z|X=x}(z)=e^{x-z}\]综上所述,\begin{equation*}
				p_{Z|X=x}(z)=\left\{
				\begin{array}{rcl}
				0 & & {z\le x}\\
				e^{x-z} & & {z>x}\\
				\end{array} \right.
				\end{equation*}
			\subsection*{15}
			\subsubsection*{(1)}
				\begin{proof}
					首先,因为$p_1(x,y)$和$p_2(x,y)$都是密度函数,所以$p_1(x,y)\ge0$且$p_2(x,y)\ge0$,故$p(x,y)\ge0$.其次,有\begin{align*}
						\int_{-\infty}^{+\infty}\int_{-\infty}^{+\infty}p(x,y)dxdy & = 0.4\int_{-\infty}^{+\infty}\int_{-\infty}^{+\infty}p_1(x,y)dxdy\\ & +0.6\int_{-\infty}^{+\infty}\int_{-\infty}^{+\infty}p_2(x,y)dxdy\\
						& = 0.4+0.6\\
						& = 1
					\end{align*}
					因此$p(x,y)$是一个密度函数.
				\end{proof}
			\subsubsection*{(2)}
				\begin{align*}
					p_X(x) & =\int_{-\infty}^{+\infty}p(x,y)dy\\
					& = \frac{0.4}{\sqrt{2\pi}\sigma_1}e^{-\frac{(x-\mu_1)^2}{2\sigma_1^2}}+\frac{0.6}{\sqrt{2\pi}\sigma_1}e^{-\frac{(x-\mu_1)^2}{2\sigma_1^2}}\\
					& = \frac{1}{\sqrt{2\pi}\sigma_1}e^{-\frac{(x-\mu_1)^2}{2\sigma_1^2}}
				\end{align*}
				同理,\[p_Y(y) = \frac{1}{\sqrt{2\pi}\sigma_2}e^{-\frac{(y-\mu_2)^2}{2\sigma_2^2}}\]
			\subsubsection*{(3)}
				即使二维随机向量的边缘分布都是正态分布,但联合分布也可能不是正态分布.
			\section*{教材习题四}
			\subsection*{8}
			\subsubsection*{(1)}
				因为\[P(A)=\frac{1}{4},P(B|A)=\frac{1}{3},P(A|B)=\frac{1}{2}\]又\[P(AB)=P(A|B)P(B)=P(B|A)P(A)\]所以可以求得\[P(B)=\frac{1}{6},P(AB)=\frac{1}{12}\]因此\begin{align*}
					& P(X=1,Y=1) = \frac{1}{12}\\
					& P(X=1,Y=0) = \frac{1}{4}-\frac{1}{12}=\frac{1}{6}\\
					& P(X=0,Y=1) = \frac{1}{6}-\frac{1}{12}=\frac{1}{12}\\
					& P(X=0,Y=0) = 1-\frac{1}{12}-\frac{1}{12}-\frac{1}{6}=\frac{2}{3}
				\end{align*}
			\subsubsection*{(2)}
				\begin{align*}
					& E(X) = \frac{1}{4}\\
					& E(Y) = \frac{1}{6}\\
					& E(X^2) = \frac{1}{4}\\
					& E(Y^2) = \frac{1}{6}\\
					& E(XY) = \frac{1}{12}
				\end{align*}
				因此\begin{align*}
					& cov(X,Y)=\frac{1}{12}-\frac{1}{24}=\frac{1}{24}\\
					& D(X) = \frac{1}{4}-\frac{1}{16}=\frac{3}{16}\\
					& D(Y) = \frac{1}{6}-\frac{1}{36}=\frac{5}{36}\\
					& \rho_{XY}=\frac{cov(X,Y)}{\sqrt{D(X)D(Y)}}=\frac{1}{\sqrt{15}}
				\end{align*}
			\subsubsection*{(3)}
				设随机变量$X$和$Y$如题干中所示,易见$X$和$Y$服从0-1分布,有\[E(X)=P(A),E(Y)=P(B),\]\[D(X)=P(A)P(\overline{A}),D(Y)=P(B)P(\overline{B})\]\[cov(X,Y)=E(XY)-E(X)E(Y)=P(AB)-P(A)P(B)\]因此即可知:\[\rho_{XY}=\rho_{AB}\]所以\[|\rho|\le1\]	
			\subsection*{13}
			\subsubsection*{(1)}
				易知\begin{align*}
					& E(X_1)=\frac{0^2+0\cdot6+6^2}{3}=12\\
					& D(X_1)=\frac{(6-0)^2}{12}=3\\
					& E(X_2)=0, D(X_2)=4\\
					& E(X_3)=5, D(X_3)=5
				\end{align*}
				\begin{align*}
					D(Y) & = D(X_1-2X_2+3X_3)\\
					& = E((X_1-2X_2+3X_3)^2)\\
					& - E^2(X_1-2X_2+3X_3)\\
					& = D(X_1)+4D(X_2)+9D(X_3)\\
					& = 3+16+45\\
					& = 64
				\end{align*}
			\subsubsection*{(2)}
				因为\begin{align*}
					& E(Y)=E(X_1)-2E(X_2)+3E(X_3)=27\\
					& E(YX_2)=E(X_1X_2-2X_2^2+3X_2X_3)=-8
				\end{align*}
				所以\begin{align*}
					\rho_{YX_2} & = \frac{E(YX_2)-E(Y)E(X_2)}{\sqrt{D(Y)D(X_2)}}\\
					& = \frac{-8}{\sqrt{64\cdot4}}\\
					& = -\frac{1}{2}
				\end{align*}
			\subsection*{16}
				因为
				\begin{align*}
					& E(X) = \int_{}^{}\int_{x^2+y^2\le1}^{}\frac{x}{\pi}dxdy = 0\\
					& E(Y) = \int_{}^{}\int_{x^2+y^2\le1}^{}\frac{y}{\pi}dxdy = 0\\
					& E(XY) = \int_{}^{}\int_{x^2+y^2\le1}^{}\frac{xy}{\pi}dxdy = 0(\textrm{奇偶性})
				\end{align*}
				所以有\[cov(X,Y)=E(XY)-E(X)E(Y)=0\]故\[\rho_{XY}=0\]所以$X$和$Y$不相关.而因为\begin{align*}
				    & p_X(x)=\int_{-\sqrt{1-x^2}}^{\sqrt{1-x^2}}\frac{1}{\pi}dy=\frac{2\sqrt{1-x^2}}{\pi}\\
				    & p_Y(y)=\int_{-\sqrt{1-y^2}}^{\sqrt{1-y^2}}\frac{1}{\pi}dx=\frac{2\sqrt{1-y^2}}{\pi}
				\end{align*}	
				此时$p(x,y)=p_X(x)p_Y(y)$不能恒成立,因此它们不独立.	
	\end{CJK}
\end{document}