\documentclass[twocolumn]{article}

\usepackage[utf8]{inputenc}
\usepackage{CJKutf8}
\usepackage{CJK}
\usepackage{algorithm}
\usepackage{algorithmic}
\usepackage{amsmath}
\usepackage{amsthm}
\usepackage{amssymb}
\usepackage{newfloat}
\usepackage{setspace}
\usepackage{tikz}
\usepackage{enumerate}
\usepackage{listings}
\usepackage{fancyhdr}
\allowdisplaybreaks[4]
\usetikzlibrary{arrows,graphs}
\usetikzlibrary{graphs}
\usetikzlibrary{graphs.standard}
\pagestyle{fancy}
\fancyhead[L]{Probability and Mathematical Statistics}
\begin{document}
	\begin{CJK}{UTF8}{gbsn}		
			\title{概率论与数理统计第11次作业}
			\author{161220049\quad 黄奕诚}
			\maketitle
			
			\section*{2}
				设总体的均值为$\mu$,方差为$\sigma^2$,则有\[\mu=\frac{1}{2}\quad \sigma^2=\frac{1}{12}\]因此样本均值$\overline{X}$的期望和方差分别为\[E(\overline{X})=\frac{1}{12}\quad D(\overline{X})=\frac{1}{12n}\]
			\section*{5}
				易知$\frac{X_i}{\sqrt{5}}]\sim N(0,1)$,于是有\[Y^2=\sum_{i=1}^{10}(\frac{X_i}{\sqrt{5}})^2\sim\chi^2(10)\]因此\[P(\sum_{i=1}^{10}X_i^2>80)=P(\frac{1}{5}\sum_{i=1}^{10}X_i^2>16)=P(Y^2>16)\]查表可知\[\chi_{0.1}^2(10)=16\]因此可求得\[P(\sum_{i=1}^{10}X_i^2>80)=0.1\]
			\section*{7}
				因为$X_1,X_2,X_3,\cdots,X_{15}$是来自正态总体$N(0,4)$的样本,所以$X_1,X_2,\cdots,X_{15}$两两相互独立,因此$\sum_{i=1}^{10}X_i^2$和$\sum_{i=11}^{15}X_i^2$独立.此时令$n_1=10,n_2=5$,则有\[Y=\frac{\sum_{i=1}^{10}X_i^2/10}{\sum_{i=11}^{15}X_i^2/5}=\frac{\sum_{i=1}^{10}(\frac{X_i}{2})^2/10}{\sum_{i=11}^{15}(\frac{X_i}{2})^2/5}=F(10,5)\]
			\section*{8}
				设$Z_i= X_i+X_{n+i}-2\mu$.由于$X_1,X_2,\cdots,X_{2n}$是来自正态总体$N(\mu,\sigma^2)$的样本,所以$Z_1,Z_2,\cdots,Z_n$两两独立,且有\[Z\sim Z_i\sim N(0,2\sigma^2)\]由此有\[\overline{Z}=\frac{1}{n}\sum_{i=1}^{n}Z_i=\overline{X}-2\mu\]所以\[Y=\sum_{i=1}^{n}(Z_i-\overline{Z})^2=\sum_{i=1}^{n}Z_i^2-2\overline{Z}\sum_{i=1}^{n}Z_i+n\overline{Z}^2\]又因为\[\sum_{i=1}^{n}Z_i=n\overline{Z}\]所以\[Y=\sum_{i=1}^{n}Z_i^2-n\overline{Z}^2\]因此\begin{align*}
					E(Y) & = \sum_{i=1}^{n}E(Z_i^2)-nE(\overline{Z}^2)\\
					& = nE(Z^2)-nE(\overline{Z}^2)\\
					& = n\cdot2\sigma^2-n(D(\overline{Z}^2)+E^2(\overline{Z}))\\
					& = 2n\sigma^2-n(\frac{2\sigma^2}{n}+0)\\
					& = 2(n-1)\sigma^2
				\end{align*}
			\section*{9}
				因为$X_1,X_2,\cdots,X_{n+1}$是来自正态总体$N(\mu,\sigma^2)$的样本,所以$X_{n+1}-\overline{X}$也服从正态分布.与此同时,\[E(X_{n+1}-\overline{X})=0\quad D(X_{n+1}-\overline{X})=\sigma^2+\frac{\sigma^2}{n}\]因此\[\frac{X_{n+1}-\overline{X}}{\sigma\sqrt{\frac{n+1}{n}}}\sim N(0,1)\]又因为\[\frac{(n-1)S^2}{\sigma^2}=\frac{\sum_{i=1}^{n}(X_i-\overline{X})^2}{\sigma^2}\sim \chi^2(n-1)\]
				所以\[\frac{X_{n+1}-\overline{X}}{S}\sqrt{\frac{n}{n+1}}=\frac{\frac{X_{n+1}-\overline{X}}{\sigma\sqrt{\frac{n+1}{n}}}}{\sqrt{\frac{(n-1)S^2/\sigma^2}{n-1}}}\sim t(n-1)\]
			\section*{10}
				\begin{enumerate}[(1)]
					\item 首先易知\[\frac{\overline{X}-12}{2/\sqrt{5}}\sim N(0,1)\]所以\[P(\overline{X}>13)=P(\frac{\overline{X}-12}{2/\sqrt{5}}>\frac{\sqrt{5}}{2})=1-\Phi(1.118)\]所以\[P(\overline{X}>13)=0.1314\]
					\item $\min_{1\le i\le5}X_i<10$等价于存在$i\in[1,5]$且$i\in N$使得$X_i<10$.设该事件为$A$,则有\begin{align*}
						P(A) & = 1-P(\overline{A})\\
						& = 1-(P(X_1\ge10))^5\\
						& = 1-(\phi(1))^5\\
						& = 0.5785
					\end{align*}
					\item 同理,设事件$B$为$\max_{1\le i\le5}X_i>15$.则有\begin{align*}
						P(B) & = 1-P(\overline{B})\\
						& = 1-(P(X_1\le15))^5\\
						& = 1-(\Phi(1.5))^2\\
						& = 0.2923
					\end{align*}
				\end{enumerate}
			\section*{11}
				不妨设第一组样本为$X_1,X_2,\cdots,X_{n_1}$,第二组样本为$Y_1,Y_2,\cdots,Y_{n_2}$.设联合样本为$Z_1,Z_2,\cdots,Z_{n_1+n_2}$.于是\[\overline{Z}=\frac{\sum_{i=1}^{n_1+n_2}Z_i}{n_1+n_2}=\frac{\sum_{i=1}^{n_1}X_i+\sum_{j=1}^{n_2}Y_j}{n_1+n_2}=\frac{n_1\overline{X}+n_2\overline{Y}}{n_1+n_2}\]并且\begin{align*}
					S_3^2 & = \frac{1}{n_1+n_2-1}\sum_{i=1}^{n_1+n_2}(Z_i-\overline{Z})^2\\
					& = \frac{\sum_{i=1}^{n_1}(X_i-\overline{Z})^2+\sum_{j=1}^{n_2}(Y_j-\overline{Z})^2}{n_1+n_2-1}
 				\end{align*}
 				又因为\[S_1^2=\frac{\sum_{i=1}^{n_1}(X_i-\overline{X})^2}{n_1-1}\quad S_2^2=\frac{\sum_{j=1}^{n_2}(Y_j-\overline{Y})^2}{n_2-1}\]
				由此可知\begin{align*}
					S_3^2 & = \frac{\sum_{i=1}^{n_1}(X_i-\overline{X}+\overline{X}-\overline{Z})^2+\sum_{j=1}^{n_2}(Y_j-\overline{Y}+\overline{Y}-\overline{Z})^2}{n_1+n_2-1}
				\end{align*}
				于是设\begin{align*}
					S_3^2 & = \frac{(n_1-1)S_1^2+(n_2-1)S_2^2+K}{n_1+n_2-1}
				\end{align*}
				其中\begin{align*}
					K & = \sum_{i=1}^{n_1}(\overline{X}-\overline{Z})^2+\sum_{i=1}^{n_1}2(X_i-\overline{X})(\overline{X}-\overline{Z})\\
					& +\sum_{j=1}^{n_2}(\overline{Y}-\overline{Z})^2+\sum_{j=1}^{n_2}2(Y_j-\overline{Y})(\overline{Y}-\overline{Z})\\
					& = n_1(\overline{X}-\overline{Z})^2+n_2(\overline{Y}-\overline{Z})^2\\
					& = \frac{n_1n_2}{n_1+n_2}(\overline{X}-\overline{Y})^2
				\end{align*}
				因此\[S_3^2=\frac{(n_1-1)S_1^2+(n_2-1)S_2^2+\frac{n_1n_2}{n_1+n_2}(\overline{X}-\overline{Y})^2}{n_1+n_2-1}\]
	\end{CJK}
\end{document}