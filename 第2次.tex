\documentclass[twocolumn]{article}

\usepackage[utf8]{inputenc}
\usepackage{CJKutf8}
\usepackage{CJK}
\usepackage{algorithm}
\usepackage{algorithmic}
\usepackage{amsmath}
\usepackage{amsthm}
\usepackage{amssymb}
\usepackage{newfloat}
\usepackage{setspace}
\usepackage{tikz}
\usepackage{enumerate}
\usepackage{listings}
\usepackage{fancyhdr}
\allowdisplaybreaks[4]
\usetikzlibrary{arrows,graphs}
\usetikzlibrary{graphs}
\usetikzlibrary{graphs.standard}
\pagestyle{fancy}
\fancyhead[L]{Probability and Mathematical Statistics}
\begin{document}
	\begin{CJK}{UTF8}{gbsn}		
			\title{概率论与数理统计第2次作业}
			\author{161220049\quad 黄奕诚}
			\maketitle
			
			\section*{1.教材习题一}
			\subsection*{(4)}
				设事件$A$为``点$A(m,n)$落入圆$x^2+y^2=19$''.样本空间共有36个样本点,而使事件$A$发生的样本点为$\{(1,1),(1,2),(1,3),(1,4),(2,1),(2,2),(2,3),(3,1),\\(3,2),(3,3),(4,1)\}$,共有11个.因此根据古典概型,$P(A)=\frac{11}{36}$.
			\subsection*{(12)}
				设事件$A$为``方程$x^2+px+q=0$有实根''.则事件$A$发生当且仅当$p^2-4q\ge0$,也即$q\le\frac{1}{4}p^2$.对于这个几何概型,$\Omega$对应区域为$\{(p,q)||p|\le1,|q|\le1\}$,而$A$对应区域为$\{(p,q)||p|\le1,|q|\le1,q\le\frac{1}{4}p^2\}$.可求出$A$对应区域的面积为\[m(A)=2+2\cdot\int_{0}^{1}\frac{1}{4}p^2dp=\frac{13}{6}\].因此\[P(A) = \frac{13}{6}\cdot\frac{1}{4}=\frac{13}{24}\].
			\subsection*{(13)}
				设事件$A$为``打成的三折能够构成三角形''.设$x,y,z$分别为打成的三折的长度,则有
				\begin{center}
				\begin{align*}
					x+y+z = 2a
				\end{align*}
				\end{center}
				当$A$发生时,有$x+y>z$且$x+z>y$且$y+z>x$.此时相当于:
				\begin{center}
				\begin{align*}
					x+y>a\\
					y<a\\
					x<a
				\end{align*}
				\end{center}
				由此根据几何概型可以计算出\[P(A)=\frac{1}{2}\cdot\frac{1}{a^2}\frac{1}{2a^2}=\frac{1}{4}\].
			\subsection*{(15)}
				由于\begin{center}
					\begin{align*}
						P(AB)=P(A|B)\cdot P(B)\\
						P(AB)=P(B|A)\cdot P(A)
					\end{align*}
				\end{center}
				可求出$P(B)=\frac{1}{6},P(AB)=\frac{1}{12}$.由此可知\begin{displaymath}
					P(A\cup B)=P(A)+P(B)-P(AB)=\frac{1}{3}
				\end{displaymath}
				因此\begin{displaymath}
					P(\overline{A}\overline{B})=1-P(A\cup B)=\frac{2}{3}
				\end{displaymath}
			\subsection*{(16)}
				设事件$A$为``取出的2件中至少有一件次品''.事件$B$为``取出的2件都是次品''.则有\[P(A)=1-\frac{C_6^2}{C_{10}^2}=\frac{2}{3}\]\[P(B)=\frac{C_4^2}{C_{10}^2}=\frac{2}{15}\]因此题中要求的概率即为\[P(B|A)=\frac{P(AB)}{P(A)}=\frac{2}{15}\cdot\frac{3}{2}=\frac{1}{5}\].
			\subsection*{(20)}
				\begin{enumerate}[(1)]
					\item 设事件$A$为``任取一盒,可以出厂''.当$A$发生时,4个抽检的元件必须都没有次品.于是\[P(A)=0.8\times1+0.1\times\frac{C_{19}^4}{C_{20}^4}+0.1\times\frac{C_{18}^4}{C_{20}^4}\approx0.943\]
					\item 设事件$A$为``一盒元件可以出厂''.事件$B$为``一盒元件无次品''.则\[P(A)\approx0.943,P(B)=0.8,P(AB)=0.8\]所以\[P(B|A)=\frac{P(AB)}{P(A)}=\frac{0.8}{0.943}=0.848\]
					
				\end{enumerate}
			\section*{2}
				C++代码如下:
				\begin{verbatim}
				#include <iostream>
				#include <cstdio>
				#include <cmath>
				#include <cstdlib>
				#include <ctime>
				#include <cstring>
				
				using namespace std;
				
				double getRandData(int min, int max)
				{
				    double m1 = (double)(rand() % 101) / 101;                     
				    min++;
				    double m2 = (double)((rand() % 
				    (max - min + 1)) + min);    
				    m2 = m2 - 1;
				    return m1 + m2;                                                            
				}
				
				int main()
				{
				    double x, y;
				    srand(time(0));
				    int trial = 100000000;
				    int in = 0;
				    for (int i = 0; i < trial; i++) {
				        x = getRandData(-2, 2);
				        y = getRandData(-2, 2);
				    if (pow(x, 2) + pow(y - pow(x, 2.0 / 3)
				        , 2) < 1.0) in++;
				    }
				    cout << (double)in/(double)trial*16.0<<endl;
				    system("pause");
				    return 0;
				}
				\end{verbatim}
				三次运行结果为1.58068、1.58109、1.58055,取其平均值为1.58077.故图形面积约为1.58
			\section*{3}
				\begin{proof}
					可以用数学归纳法证明.
					\begin{itemize}
						\item 当盒子中存在2个球时,白球数目必为1;当盒子中存在3个球时($n=3$),白球数目为1和2的概率均为$\frac{1}{2}$.
						\item 假设当$n=k,k\in N^*,k\ge2$时结论成立,也即当盒子中存在$k$个球时,盒子里白球的数目等可能地为1到$k-1$的某个数.则设此时白球个数为$t$,有\[P_k(t=1)=P_k(t=2)=\cdots=P_k(t=k-1)=\frac{1}{k-1}\]不妨简写为\[P_k(1)=P_k(2)=\cdots=P_k(k-1)=\frac{1}{k-1}\] 当共有$k+1$个球时,记此时白球个数为$t$.有\[P_{k+1}(t)=P_k(t-1)\cdot\frac{t-1}{k}+P_k(t)\cdot\frac{k-t}{k}=\frac{1}{k}\]因此此时盒子里白球的数目等可能地为1到$k$的某个数.
						\item 综上所述,当盒子中存在$n$个球时,白球个数等可能地分布在1到$n-1$.
						
					\end{itemize}
				\end{proof}
			\section*{4}
				点数和是6的倍数的概率总为$\frac{1}{6}$.
				\begin{proof}
					设抛掷一枚均匀骰子$k$次,设其点数和为6的倍数的概率为$P_k(A)$,点数和不为6的倍数的概率为$P_k(\overline{A})$.则设抛掷$k+1$次时,点数和为$t$,有\[P_{k+1}(t)=\frac{1}{6}P_{k}(t-1)+\frac{1}{6}P_{k}(t-2)+\cdots+\frac{1}{6}P_{k}(t-6)\]也即\[P_{k+1}(t)=\frac{1}{6}(P_k(t-1)+P_k(t-2)+\cdots+P_k(t-6))\]所以抛掷第$k+1$次时,无论前$k$次抛掷出的点数和是否是6的倍数,其此时的点数和是6的倍数的概率都是在先前$k$次抛掷条件下的$\frac{1}{6}$.也即\[P_{k+1}(A)=\frac{1}{6}P(A)+\frac{1}{6}P(\overline{A})=\frac{1}{6}\]
				\end{proof}
	\end{CJK}
\end{document}